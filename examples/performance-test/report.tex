\documentclass[a4paper]{article}

\usepackage{tikz}
\usepackage{amssymb}

\newcommand\KeY{Ke\kern-.1emY}

\title{\KeY\ Performance Test Report}
\author{Daniel Bruns}

\begin{document}

\maketitle

\section{Test Series}
\begin{itemize}
  \item Inc: increase just one field 
  \item Inc2: alternatively increase two fields
  \item Contract: use an operation contract with
		concrete frame
  \item GhostFrame: use an operation contract
		with an abstract frame stored in
		a ghost field
  \item Model field: use an operation contract
		with an abstract frame stored in
		a model field (static depends clause)
  \item Dynamic: use an operation contract
		with an abstract frame stored in
		a model field (self-framing depends clause)
  \item AccessChain: same as Contract, but assign field
	$\mathit{this}.(a.)^nx$
\end{itemize}

All figures of proof steps count One-Step-Simplifier applications
as only one step.

\section{Empirical Results}
\subsection{Basic Computation Time Growth}
Computation time grows at least quadratically in the
size of the input program.
In \texttt{Inc}, the number of proof steps grows
exactly linear, but computation times grows (effectively) linear in the number of proof steps.

Proof size vs.\ computation time (in seconds):\\
\begin{tikzpicture}
  % grid, labels
  \draw[->] (0,-1) -- (7,-1);
  \draw[->] (0,-1) -- (0,4.5);
  \foreach \x in {0,1,2,3,4,5,6} {
    \draw (\x,-1) -- (\x,-1.1);
    \node at (\x,-1.5) {$10^{\x}$};
  }
  \foreach \y in {-1,0,1,2,3,4} {
    \draw (0,\y) -- (-.1,\y);
    \node at (-.5,\y) {$10^{\y}$};
  }

  % data
  \node at (2.33,-.22) {$\bullet$};
  \node at (2.59,-.52) {$\bullet$};
  \node at (3.24,.26) {$\bullet$};
  \node at (3.54,.61) {$\bullet$};
  \node at (4.23,1.6) {$\bullet$};
  \node at (4.53,2.03) {$\bullet$};
  \node at (4.83,2.67) {$\bullet$};
  \node at (5,3.16) {$\bullet$};
  \node at (5.23,3.662) {$\bullet$};

  \node at (3.143,1.503) {\textcolor{blue}{$\diamond$}};

  % extrapolate linear vector
  \draw[color=red] (2.7,-.8) -- (5.4,3.6);
\end{tikzpicture}

Comparison values: TPTP PUZ031 (\textcolor{blue}{$\diamond$})

\subsection{Large Sequences}
Large sequences (i.e., containing many formulae/terms) seem to slow
down \KeY\ significantly.

\subsection{Comparison of Framing Techniques}
Reasoning about location sets is one of the major sources of effort.
Static framing leads to a linear sequent size growth
(containing information about different heaps).

In abstract framing large heap terms occur.
They appear when simplifying a select on an anonymized heap,
introducing if-then-else statements.
This leads to an exponential growth of heap terms.

\begin{itemize}
  \item footprint is a model field (static dependency) 
	[\textcolor{green}{$\bullet$}]
  \item footprint is a ghost field (dynamic dependency)
	[\textcolor{blue}{$\ast$}]
  \item footprint is a model field (dynamic dependency)
	[\textcolor{red}{$\diamond$}]
\end{itemize}

Problem size (statements) vs.\ time (seconds):\\
\begin{tikzpicture}
  % grid, labels
  \draw[->] (0,-1) -- (10.5,-1);
  \draw[->] (0,-1) -- (0,4);
  \foreach \x in {1,2,3,4,5,6,7,8,9,10} {
    \draw (\x,-1) -- (\x,-1.1);
    \node at (\x,-1.5) {${\x}$};
  }
  \foreach \y in {-1,0,1,2,3} {
    \draw (0,\y) -- (-.1,\y);
    \node at (-.5,\y) {$10^{\y}$};
  }

  % data
  \node at (1,.119) {\textcolor{red}{$\diamond$}};
  \node at (2,.034) {\textcolor{red}{$\diamond$}};
  \node at (4,.99) {\textcolor{red}{$\diamond$}};

  \node at (1,-.107) {\textcolor{green}{$\bullet$}};
  \node at (2,-.206) {\textcolor{green}{$\bullet$}};
  \node at (4,.131) {\textcolor{green}{$\bullet$}};
  \node at (8,1.661) {\textcolor{green}{$\bullet$}};
  \node at (10,2.847) {\textcolor{green}{$\bullet$}};

  \node at (1,.027) {\textcolor{blue}{$\ast$}};
  \node at (2,-.111) {\textcolor{blue}{$\ast$}};
  \node at (4,.001) {\textcolor{blue}{$\ast$}};
  \node at (8,1.32) {\textcolor{blue}{$\ast$}};
  \node at (10,2.12) {\textcolor{blue}{$\ast$}};
\end{tikzpicture}

\subsection{Depth of select-Terms}
Length access chains (e.g., $\mathit{this.next.next.next}$)
contributes to exponential growth of prover time.

\subsection{Comparison of \KeY\ Versions}
on contract benchmark (static frame)

\begin{itemize}
  \item Release 1.6.5 (\textcolor{green}{$\bullet$})
  \item Release 2.0.0 (\textcolor{red}{$\diamond$})
  \item Development branch \texttt{schebenHeapTermSimplification}
		(\textcolor{blue}{$\ast$})
\end{itemize}

\begin{tikzpicture}
  % grid, labels
  \draw[->] (-.5,-1) -- (9,-1);
  \draw[->] (-.5,-1) -- (-.5,4.5);
  \foreach \x in {0,1,2,3} {
    \draw (2.5*\x,-1) -- (2.5*\x,-1.1);
    \node at (2.5*\x,-1.5) {$10^{\x}$};
  }
  \foreach \y in {-1,0,1,2,3,4} {
    \draw (-.5,\y) -- (-.6,\y);
    \node at (-1,\y) {$10^{\y}$};
  }

  % data
  \node at (0,-.155) {\textcolor{green}{$\bullet$}};
  \node at (2.5*0.3,-.523) {\textcolor{green}{$\bullet$}};
  \node at (2.5*1,-.155) {\textcolor{green}{$\bullet$}};
  \node at (2.5*1.3,-.097) {\textcolor{green}{$\bullet$}};
  \node at (2.5*1.6,.462) {\textcolor{green}{$\bullet$}};
  \node at (2.5*2,.724) {\textcolor{green}{$\bullet$}};
  \node at (2.5*2.3,1.318) {\textcolor{green}{$\bullet$}};

  \node at (2.5*0.3,-1) {\textcolor{red}{$\diamond$}};
  \node at (2.5*1,-.046) {\textcolor{red}{$\diamond$}};
  \node at (2.5*1.3,.544) {\textcolor{red}{$\diamond$}};
  \node at (2.5*1.6,1.387) {\textcolor{red}{$\diamond$}};
  \node at (2.5*2,2.58) {\textcolor{red}{$\diamond$}};
  \node at (2.5*2.3,3.745) {\textcolor{red}{$\diamond$}};

  \node at (0,-.009) {\textcolor{blue}{$\ast$}};
  \node at (2.5*0.3,-.216) {\textcolor{blue}{$\ast$}};
  \node at (2.5*1,.128) {\textcolor{blue}{$\ast$}};
  \node at (2.5*1.3,.551) {\textcolor{blue}{$\ast$}};
  \node at (2.5*1.6,1.125) {\textcolor{blue}{$\ast$}};
  \node at (2.5*2,2.082) {\textcolor{blue}{$\ast$}};
  \node at (2.5*2.3,3.011) {\textcolor{blue}{$\ast$}};
\end{tikzpicture}


\section{Overall Complexity}
\begin{tabular}{|l|l|l|l|l|} \hline
Test & Heap & Seq & Steps & Time \\ \hline
\texttt{Inc}& $\Theta(1)$ & $\Theta(1)$ & $\Theta(n)$ & $\Omega(n^2)$ \\
\texttt{Inc2} & $\Theta(1)$ & $\Theta(1)$ & $\Theta(n)$ & $\Omega(n^2)$ \\
\texttt{Contract} & $\Theta(1)$ & $\Theta(n)$ 
	& $\Omega(n^2)$ & $\Omega(n^4)$ \\
\texttt{ModelField} & $\Omega(2^n)$ & ? & ? & ? \\
\hline
\end{tabular}

\section{Solution Suggestions}

\paragraph{Flat heap terms.}
ad heap term growth:
	add taclets to store equalities instead of contructing large terms
	(on branch \texttt{schebenHeapTermSimplification}

Works well on large instances of Contract (see above):
 nearly linear growth in logarithmical scale
\end{document}
